\documentclass[../../dd.tex]{subfiles}

\begin{document}
\chapter{Introduction}
\section{Purpose}
    In this section we will briefly introduce the application.
    The application is developed for
    the course of \textit{Design and Implementation of Mobile Applications} of professor Luciano Baresi
    at Politecnico di Milano.
    The goal of the course is to design and implement a mobile application on
    a platform of our choice.
    This document illustrates the decisions we made in
    order to accomplish this goal.
    \\

    This Software Design Document is a document that provides documentation that will be used as
    overall guidance to the architecture of the software project.
    Moreover, I will provide documentation of the software design of the
    project, including use case models, class and sequence diagrams
    \\

    The purpose of this document is to provide a full description of the design of \textit{Baddy},
    a cross-platform application, providing insights into the structure and design of each component.

    \section{Scope}
    The idea of this application comes from a need for the elders to find someone that can take care of them,
    if they do not have children or they are not available.
    We are sure that the next generation of old people will be able to use applications smoothly without any trouble.
    We will now briefly describe all the functionalities.
    \\

    The user will be able to register and search for a person who is going to take care of him.
    The caregiver too needs to register and set its city, name and phone number in order to be reached by all
    the other users.
    Elders can also write reviews and rate the caregiver.
    Caregivers can update their profile with additional information like age, birth, nationality, gender and set their
    state to be available for a job or not.

    \section{Stakeholders}
    The main stakeholders of this app are the elders and people that wants to earn something extra as caregiver,
    but not limited to children of the elder that want to find someone when they are absent for example.
    \\

    Moreover, we have to consider, as a stakeholder, professor Luciano Baresi that
    holds the \textit{Design and Implementation of Mobile Applications} course and Giovanni Quattrocchi,
    teaching assistant of the course.

    \section{Time Constraints}
    We have no precise deadline for this application, our idea is to ultimate it for one of the 2 calls in Jan/Feb.
    We already know the technologies to develop the application so we have to spend less time for that part.
    We are starting the development in November together with this DD, then we will do all the necessary testing.

    \section{Overview}
    This document is structured as follows:
    \begin{enumerate}
        \item \textbf{Introduction.} A general introduction of the Design and Technology Document.
        It aims giving general but exhaustive information
        about what this document is going explain.
        \item \textbf{General Overview.} A general overview of the project.
        In this section, the reader could find the core features of the application and
        the requirements of the system.
        \item \textbf{Architectural Design.} This section contains an overview
        of the high-level components of the system and then a more detailed
        description of three architecture views: Component view, Deployment
        View and Runtime View.
        Finally, it shows the Component Interfaces
        and the chosen architecture styles and patterns.
        \item \textbf{User Interface Design.} This section contains the screenshots
        of the application with some comments to give to the reader a general
        overview of the user interfaces.
        \item \textbf{Frameworks, External services and Libraries.} This section
        aims to explain the main frameworks, external services and libraries
        used pointing out their advantages.
        \item \textbf{Test Cases.} This section identifies test cases performed reporting
        their results.
        \item \textbf{Effort and Cost Estimation.} A summary of worked time and
        cost estimation of the work.
        \item \textbf{Future Works.} This section contains some new features that,
        in future, could be add to \textit{Baddy} or the Application Server.
    \end{enumerate}

    At the end, there is an Appendix where software and tools used are reported.
\end{document}
