\documentclass[../../dd.tex]{subfiles}

% Document
\begin{document}
    \chapter{General Overview}
    \section{Concept}
    \textit{Baddy} is a cross-platform application that allows people to find a caregiver for the elders.
    The application requires both end users and caregivers to authenticate with their email, filling a customized form
    based on the role of the user.
    Those data are mandatory in order to use the application.
    The main components of the project are:
    \begin{itemize}
        \item A client that queries the database to fetch data of users, but also send data to it,
        like pictures or new user data.
        \item A server that listens to requests and links the client to the database with all the needed logic, like
        the authentication part for example.
    \end{itemize}
    \section{Core Features}
    This section details a list of the core functionalities of the application. The features are divided by the screen where they are present, allowing the reader to easily understand them and indicating where they could be found in the application.
    \subsection{Splash Screen}
    This is a welcome page screen that pops up only for the very first time a new user opens the application.
    It briefly describes all the functionalities contained in the app.

    \subsection{Authentication Screens}
    \begin{itemize}
        \item Allows the user to authenticate using his/her email and password.
        \item Allows the user to register his/her-self in the application.
    \end{itemize}

    \subsection{Homepage}
    \begin{itemize}
        \item Allows the user to search for a caregiver, by specifying the city.
        \item Shows all the information of the caregivers, including name, city, price/h and average rating
        \item Allows to navigate to the profile page if logged in as caregiver
        \item Allows to log out the user.
    \end{itemize}

    \subsection{Update Profile Screen}
    \begin{itemize}
        \item Allows the caregiver to change his personal information, such as birth, gender, nationality,
        city, availability, name, surname, phone number.
        \item Allows the caregiver to update his profile photo.
    \end{itemize}

    \subsection{Profile Screen}
    \begin{itemize}
        \item Allows users to check for specific information about a caregiver, such as birth, gender, nationality,
        city, availability, name, surname, phone number.
        \item Allows users to check for reviews written by other users about the specific caregiver.
        \item Allows users to write a review about the specific caregiver.
    \end{itemize}

    \section{Functional Requirements}
    \subsection{General Requirements}
    \begin{itemize}
        \item The app can be used by everyone who needs to find a caregiver, and also all caregivers can use
        tha application in order to find new clients.
        \item The app needs user auth, so it has also to provide a registration form.
        \item The app needs internet connection in order to communicate with the server.
        \item The server must accept requests from the client app.
    \end{itemize}

    \subsection{Splash Requirements}
    \begin{itemize}
        \item Splash screen should be visible only the first time the user opens the application.
        \item Splash screen should contain a text followed by an image that explains the purpose of the app.
        \item A swipe left/right should be present in order to switch between 3 different splash screens.
    \end{itemize}

    \subsection{Authentication Requirements}
    \begin{itemize}
        \item Login screen should be accessible only to users not currently logged in.
        \item Authentication screens should allow users to register or login using
        email and password.
        If the user wants to sign up as a caregiver, additional information is required: phone number and city.
        \item Authentication screens should redirect the user to the Homepage
        if the authentication is successful.
    \end{itemize}

    \subsection{Homepage Requirements}
    \begin{itemize}
        \item \textit{The homepage} should contain a search box for the city, and a list of caregivers.
        \item Each list tile should contain name, surname, city, photo, price/h, and average rating of the caregiver.
        \item As soon as the user type something in the search box, list tiles that do not match the current string should disappear.
        \item Whenever the user taps on the list tile, the \textit{Profile Screen} should appear.
        \item The app bar should contain the name of the logged user plus 2 action buttons, the first to update the profile
        (only if logged in as caregiver, otherwise is hidden), and the second to logout.
    \end{itemize}

    \subsection{Update Profile Requirements}
    \begin{itemize}
        \item \textit{Update Profile Page} should be visible only by the caregiver.
        \item \textit{Update Profile Page} should allow the caregiver to update his information, so that all the other users can see the changes.
        \item The date should be inserted in the right format.
        \item Gender can be chosen between 3 options only: male, female, transgender.
        \item The caregiver must be adult in order to use the application, so the birthdate must be compliant with that.
        \item The caregiver must be able to update his photo, either by using the camera or the gallery.
    \end{itemize}

    \subsection{Profile Requirements}
    \begin{itemize}
        \item \textit{The profile} should display the photo of the selected caregiver.
        \item \textit{The profile} should display all the information that can be inserted in the \textit{Update Profile Screen}.
        \item A bottom bar must be visible in order to swap tabs, the first tab contains information about the caregiver, the second tab
        contains a list of reviews written for the current caregiver, and the third one contains a form to write the review with the possibility
        to choose how many stars give to the caregiver.
        \item Caregiver are not allowed to write reviews.
    \end{itemize}

    \section{Non-Functional Requirements}
    These are the non-functional requirements that specify criteria that can be
    used to judge the operations of the application.
    \begin{itemize}
        \item \textbf{Availability} the services must be always up and running.
        In case of failure it must be restored as soon as possible.
        \item \textbf{Extensibility} the application was developed trying to keep a simple
        structure in order to allow further extensions easily.
        \item \textbf{Maintainability} the code must be clear, readable and with explicative
        comments to allow future maintenance.
        \item \textbf{Eye catching UI} nowadays is very important to have a cleaned
        and simple design.
        The application was developed following the Apple
        Human Interface Guidelines.
        \item \textbf{Portability} cross-platform implementation allows to use the application both in Android and iOS systems;
        \item \textbf{Reliability} all the data must be trusted, and they could not be modified
        by anyone.
        The remote database must be protected and the connection
        between client and server must be handled with an HTTPS protocol.
        \item \textbf{Scalability} the system (considering client and server) should always
        be available to be used.
        System failures and server-side crashes must be avoided;
        \item \textbf{Usability} the application was developed to make the user interface as
        simple as possible keeping all the functionalities needed to provide the
        best user experience.

    \end{itemize}


\end{document}
